\newpage
\section{Problema}
La ditta C \& B co. produce chitarre e bassi partendo da un set di componenti prestabiliti e assemblandoli insieme per produrre gli strumenti.
 
Ogni componente esiste in due versioni: per chitarra e per basso; nonostante abbiano lo stesso nome, infatti, non si possono montare i componenti per chitarra su un basso e viceversa, tranne in alcuni casi specificati in seguito. Nella seguente tabella sono indicati nello specifico i componenti, i rispettivi prezzi e, per questioni di mercato, il numero massimo di ognuno che può essere acquistato dall'azienda in un anno.

\begin{table}[htbp]
\resizebox{\textwidth}{!}{
\begin{tabular}{|l|l|l|l|l|}
\hline
                  & \multicolumn{2}{c|}{Chitarra}            & \multicolumn{2}{c|}{Basso}               \\ \hline
                  & \textbf{Quantità acquistabile} & \textbf{Prezzo (\euro /pezzo)} & \textbf{Quantità acquistabile} & \textbf{Prezzo (\euro /pezzo)} \\ \hline
\textbf{Chiavette}         & 66636                       & 3.5                 & 66834                       & 4.5                 \\ \hline
\textbf{Capotasto}         & 11103                       & 2                 	 & 11231                       & 2                 \\ \hline
\textbf{Manico}            & 11071                       & 48                 & 11097                       & 44                 \\ \hline
\textbf{Tastiera}          & 11071                       & 16.5                 & 11097                       & 15.3                 \\ \hline
\textbf{Tasti}             & 225000                      & 0.5                 & 223211                      & 0.7                 \\ \hline
\textbf{Segnatasti}        & 88920                       & 0.95                 & 88135                       & 1.05                 \\ \hline
\textbf{Corpo}             & 11903                       & 51                 & 11081                       & 49                 \\ \hline
\textbf{Battipenna}        & 22304                       & 8                 & 22202                       & 7                 \\ \hline
\textbf{Ponte}             & 11106                       & 11                 & 11239                       & 11                 \\ \hline
\textbf{Pick-up}           & 33211                       & 42                 & 22207                       & 47.5                 \\ \hline
\textbf{Selettore pick-up} & 22000                       & 9                 & 22000                       & 7                 \\ \hline
\textbf{Potenziometri}     & 33522                       & 16.5                 & 33633                       & 15.2                 \\ \hline
\textbf{Jack d'uscita}     & 11240                       & 3                 & 11230                       & 3                 \\ \hline
\textbf{Truss-rod}         & 11071                       & 24                 & 11097                       & 28                 \\ \hline
\end{tabular}}
\end{table}

La ditta produce quattro modelli di chitarre e due modelli di basso. Ogni modello necessita di una determinata quantità di componenti per poter essere realizzato; i dati sono riassunti nelle seguenti tabelle:
\\ \\
\textbf{Chitarre}
\begin{table}[htbp]
\begin{center}
\resizebox{\textwidth}{!}{
\begin{tabular}{|l|l|l|l|l|l|l|l|l|l|l|l|l|l|l|}
\hline
\textbf{}       & \textbf{Chiav.} & \textbf{Cap.} & \textbf{Man.} & \textbf{Tastiere} & \textbf{Tasti} & \textbf{Segnat.} & \textbf{Corpi} & \textbf{Batt.} & \textbf{Ponti} & \textbf{P.-U.} & \textbf{Selett. P.-U.} & \textbf{Pot.} & \textbf{Jack} & \textbf{T.-R.} \\ \hline
\textbf{LP}     & 6               & 1             & 1             & 1                 & 26             & 11                & 1              & 0              & 2              & 2              & 1                      & 4             & 1             & 1              \\ \hline
\textbf{Strato} & 7               & 1             & 1             & 1                 & 19             & 8                & 1              & 1              & 1              & 3              & 1                      & 2             & 1             & 1              \\ \hline
\textbf{Tele}   & 6               & 1             & 1             & 1                 & 22             & 9                & 1              & 1              & 1              & 1              & 1                      & 2             & 1             & 1              \\ \hline
\textbf{EDS}    & 18              & 2             & 2             & 2                 & 43             & 15               & 2              & 2              & 3              & 5              & 2                      & 4             & 1             & 2              \\ \hline
\end{tabular}}
\end{center}
\end{table}

\textbf{Bassi}

\begin{table}[htbp]
\resizebox{\textwidth}{!}{
\begin{tabular}{|l|l|l|l|l|l|l|l|l|l|l|l|l|l|l|}
\hline
\textbf{}            & \textbf{Chiav.} & \textbf{Cap.} & \textbf{Man.} & \textbf{Tastiere} & \textbf{Tasti} & \textbf{Segnat.} & \textbf{Corpi} & \textbf{Batt.} & \textbf{Ponti} & \textbf{P.-U.} & \textbf{Selett. P.-U.} & \textbf{Pot.} & \textbf{Jack} & \textbf{T.-R.} \\ \hline
\textbf{Mustang}     & 4               & 1             & 1             & 1                 & 19             & 8                & 1              & 1              & 1              & 1              & 1                      & 1             & 1             & 1              \\ \hline
\textbf{Thunderbird} & 6               & 1             & 1             & 1                 & 21             & 9                & 1              & 1              & 2              & 2              & 1                      & 3             & 1             & 2              \\ \hline
\end{tabular}}
\end{table}

Come già detto, esistono delle eccezioni: si può infatti usare un selettore pick-up per chitarra su un basso (e viceversa) aggiungendo un adattatore con costo aggiuntivo di 1\euro , e/o un potenziometro per chitarra su un basso (e viceversa) aggiungendo un modulatore con costo aggiuntivo di 2\euro . \\
\newpage
I prezzi con cui gli strumenti vengono immessi sul mercato sono riportati nella seguente tabella:

\begin{table}[htbp]
\begin{center}
\begin{tabular}{|l|c|}
\hline
\textbf{Strumento} & \textbf{Prezzo di vendita(\euro )} \\ \hline
\textbf{LP}                 &  624                          \\ \hline
\textbf{Strato}             &  559                          \\ \hline
\textbf{Tele}               &  476                          \\ \hline
\textbf{EDS}                &  1169                          \\ \hline
\textbf{Mustang}            &  309                          \\ \hline
\textbf{Thunderbird}        &  449                          \\ \hline
\end{tabular}
\end{center}
\end{table}

Per la produzione degli strumenti, l'azienda possiede tre stabilimenti produttivi A, B e C, ognuno con una quantità di ore di manodopera prestabilita; questa quantità è rispettivamente 20000, 30000 e 40000 ore. Ogni ora costa all'azienda 10\euro . \\
Ogni strumento può essere prodotto in ognuno dei tre stabilimenti, ma a causa della diversità di mezzi produttivi a disposizione il tempo per produrre un modello in uno stabilimento non è necessariamente lo stesso che si avrebbe in un altro stabilimento; in tabella sono riassunti questi dati.

\begin{table}[htbp]
\begin{center}
\resizebox{\textwidth}{!}{
\begin{tabular}{|l|l|l|l|}
\hline
\textbf{Modello}     & \textbf{Manodopera stab. A (h)} & \textbf{Manodopera stab. B (h)} & \textbf{Manodopera stab. C (h)} \\ \hline
\textbf{LP}          &  1.2                           &  4.5                           & 5.5                            \\ \hline
\textbf{Strato}      &  2.5                           &  3.7                           & 5.6                            \\ \hline
\textbf{Tele}        &  3.0                           &  4.3                           & 6.6                            \\ \hline
\textbf{EDS}         &  3.8                           &  3.5                           & 4.1                            \\ \hline
\textbf{Mustang}     &  2.6                           &  3.4                           & 3.0                            \\ \hline
\textbf{Thunderbird} &  2.9                           &  4.0                           & 5.9                            \\ \hline
\end{tabular}}
\end{center}
\end{table}

Gli strumenti, inoltre, possono essere modificati dalla ditta su richiesta del cliente. Ogni modifica ha lo stesso costo e lo stesso uso di manodopera per ogni modello in ogni stabilimento; il costo per modello, la quantità di manodopera e il ricavo per l'azienda sono di seguito riportati:

\begin{table}[htbp]
\begin{center}
\resizebox{\textwidth}{!}{
\begin{tabular}{|l|l|l|l|}
\hline
\textbf{Modello}     & \textbf{Costo modifica(\euro)} & \textbf{Manodopera (h)} & \textbf{Aumento di prezzo (\euro)} \\ \hline
\textbf{LP}          & 34                        & 0.5                    &	45				         \\ \hline
\textbf{Strato}      & 33                        & 0.9                    &	41				         \\ \hline
\textbf{Tele}        & 32                        & 0.6                    &	55				         \\ \hline
\textbf{EDS}         & 39                        & 0.4                    &	49				         \\ \hline
\textbf{Mustang}     & 30                        & 0.6                    &	43				         \\ \hline
\textbf{Thunderbird} & 29                        & 0.5                    &	42						  \\ \hline     
\end{tabular}}
\end{center}
\end{table}

Si richiede di calcolare il mix ottimo di produzione dei vari strumenti, al fine di massimizzare il profitto dell'azienda.